\subsection{Decision Node: the Probabilistic Model}
The Decision Node is the central block of the implemented system and the main focus of this project. The role of the Decision Node is to interpret the information coming from the Kinect Node and decide which command should be sent to the Action Node.

The decision process we implemented is based on a machine learning approach. Based on the assumptions made in the Introduction we chose a system that would learn the most common sequences of gestures of a user and their appropriate response by the robot. 

For this approach we needed a way to generate labeled data for our classifier, so what we implemented was getting the outputs of the kinect node and saving them to a dataset with the labels we gave by hand.

The samples taken from the kinect were based on interactions we defined, and that represented what we thought were normal human attempts to interact with a robot. Some examples of interactions are the following:

\begin{itemize}
\item User moves around; waves; goes towards the camera; hands an object.
\item User waves; calls; gives an high-five.
\item User calls; goes away.
\end{itemize}

Each dataset sample represented the last five gestures by the user and corresponded to the output of the \textit{feature\_publisher} node. The structure of each sample is the following:

\vspace*{4pt}
$\big[Distance, [Gesture, Timestamp, Velocity_x]_1,...$\\$
[Gesture, Timestamp, Velocity_x]_5\big]$
\vspace*{4pt}

To every structure arriving the Training Set, a label is given. This label represents the action that Adele should take as a response to that particular human behavior.

To explain the idea behind this labeling let's use the waving gesture as an example. The moment immediately before the user starts waving, the information coming from the Kinect is already labeled as a "Waving", so that Adele knows that when the user is at that particular distance and moves his arm at that particular velocity and with that amplitude of movement, he is waving at her and she must react properly (waving back). Ideally the features read by the Kinect node are sufficient to distinguish the "waving" from any other gesture and therefore Adele starts reacting before she sees the complete movement.

As mentioned before, the labels that were given, depend only on the person's movement/gesture and represent Adele's reaction. Table \color{red}\ref{tab:expected_response} \color{black} shows the dependencies between both aspects.
\begin{table}[!h]
\centering
\caption{Expected response from Adele}
\label{tab:expected_response}
\begin{tabular}{|c|c|}
\hline
\textbf{Person's gesture/movement} & \textbf{Adele's response} \\ \hline
Waving                           & Wave                                 \\
Handing an object                & Grab                            		\\
Calling                          & Go meet user                         \\
Go away                          & Return to home                       \\
High-Five                        & Low-Five                             \\
Low-Five                         & High-Five                            \\
Other	                           & NOP                                  \\ \hline
\end{tabular}
\end{table}

The next step in the implementation of the Decision Node concerns Adele's learning. The Training Set was divided into two parts, one containing 80\% and the other with the remaining 20\%, the latter being called the validation set.

Regarding how the validation is done, there were some different machine learning algorithms that were tested, such as: SVM, Linear SVM, Logistic Regression, 3 Nearest Neighbours, AdaBoost and Stochastic Gradient Descent. The algorithm that presented the best performance using 10-fold cross-validation was the SVM and therefore that was the chosen one.

There is a probability associated with each action performed by Adele. Based on the user movement she captures using the kinect, she labels each set of 5 gestures detected according with the program previously trained. Each label represents an action to be performed by Adele. To filter the most probable action, based on the user initial movements of a gesture, we set a threshold from which we consider to have a degree of certainty that the action which the probability beats the threshold is the one to be performed. 

Probabilities for each action are incremented based on the number of times Adele labels a set of 5 gestures recognized with that desired action. Furthermore, the intiation of these probabilities is not random. 

Human knowledge was used to help the system to predict which gesture is more probable to be done after a previous gesture. 

Bearing that in mind we designed a probability graph with the probabilities for the sequence of actions we found more natural Adele to perform given previous actions. These probabilities are assigned to each possible action after the degree of certainty about the previous one be enough to beat the threshold. In the beginnig we assume the system is performing the NOP action.

The graph is represented as table showing the probability of doing a particular action after other action. This information is shown in table 3.

\begin{table}[]
\centering
\caption{My caption}
\label{my-label}
\begin{tabular}{cccccccc}
Previous|Next                        & \textbf{NOP} & \textbf{Wave} & \textbf{H5} & \textbf{L5} & \textbf{Go} & \textbf{Grab} & \textbf{Return} \\ \cline{2-8} 
\multicolumn{1}{c|}{\textbf{NOP}}    & 1/7          & 1/7           & 1/7         & 1/14        & 1/7         & 3/14          & 1/7             \\
\multicolumn{1}{c|}{\textbf{Wave}}   & 1/7          & 0             & 0           & 1/7         & 3/7         & 2/7           & 0               \\
\multicolumn{1}{c|}{\textbf{H5}}     & 1/6          & 1/6           & 0           & 1/6         & 1/6         & 1/6           & 1/6             \\
\multicolumn{1}{c|}{\textbf{L5}}     & 2/7          & 0             & 0           & 0           & 0           & 2/7           & 3/7             \\
\multicolumn{1}{c|}{\textbf{Go}}     & 1/14         & 0             & 1/7         & 1/7         & 0           & 3/7           & 3/14            \\
\multicolumn{1}{c|}{\textbf{Grab}}   & 2/7          & 0             & 1/7         & 1/7         & 0           & 0             & 3/7             \\
\multicolumn{1}{c|}{\textbf{Return}} & 1/6          & 1/6           & 1/6         & 1/6         & 1/6         & 1/6           & 0              
\end{tabular}

\end{table}