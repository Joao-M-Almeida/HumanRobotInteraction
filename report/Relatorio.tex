\input{headers}
\begin{document}
%
% paper title
% Titles are generally capitalized except for words such as a, an, and, as,
% at, but, by, for, in, nor, of, on, or, the, to and up, which are usually
% not capitalized unless they are the first or last word of the title.
% Linebreaks \\ can be used within to get better formatting as desired.
% Do not put math or special symbols in the title.
\title{Human-Robot Interaction by Prediction of Human Actions\\ Autonomous Systems - Instituto Superior T\'{e}cnico}


% author names and affiliations
% use a multiple column layout for up to three different
% affiliations
\author{\IEEEauthorblockN{João Rocha e Melo}
\IEEEauthorblockA{70831}
\and
\IEEEauthorblockN{Filipe Costa}
\IEEEauthorblockA{73055}
\and
\IEEEauthorblockN{Diogo Maximino}
\IEEEauthorblockA{73110}
\and
\IEEEauthorblockN{João Almeida}
\IEEEauthorblockA{73198}
\and
\IEEEauthorblockN{João Severino}
\IEEEauthorblockA{73608}}

\maketitle

% As a general rule, do not put math, special symbols or citations
% in the abstract
\begin{abstract}
Within the framework of this project, in order to be able to have an interaction between the human user and the robot it was used a Kinect camera, the Scout base robot and the Katana arm.\\
The Kinect was used for the system to learn informations about the outer world, in this case to learn about positions and movements of the user, with help of a skeleton-tracker software.\\
For controlling the Scout a simple controller was created that allows the robot to move forward and backwards.\\
The Katana represents one big part of the robot interaction, given that the robotic arm has a lot of movement possibilities being therefore possible to replicate human-like movements/behaviours.\\
Between the input information coming from the Kinect and the output provided by the Scout and Katana a machine learning algorithm was developed to introduce the prevision part of the project to the whole system.
\end{abstract}

% no keywords


% For peer review papers, you can put extra information on the cover
% page as needed:
% \ifCLASSOPTIONpeerreview
% \begin{center} \bfseries EDICS Category: 3-BBND \end{center}
% \fi
%
% For peerreview papers, this IEEEtran command inserts a page break and
% creates the second title. It will be ignored for other modes.
\IEEEpeerreviewmaketitle

\section{Introduction}
% no \IEEEPARstart
The main goal of this project is to make a mobile robot display human-aware behaviours, based on the predicted behaviours of the human the robot is interacting with. For this project we worked with the robotic arm Katana, the mobile robotic base Scout and the Kinect camera. These 3 components build our system, which we decided to call \textit{Adele}.

To achieve this goal there are some issues that need to be solved at first, such as: learning how to work with the software ROS (Robot Operating System), making the robot mobile, collecting data from the camera, interpret this data, create an algorithm to make a more human-like response from the robotic system and finally integrate all these parts of the project.

It is essential to have a good understanding of the ROS software, because in is the foundation of the project. ROS allows to use packages for the robots and camera, as well as creates an interface that facilitates the communication between all the system parts.

Considering the second point mentioned - making the robot mobile - there is the need to use already created ROS packages that give us drivers both for Scout as for Katana. Furthermore the movements that both robots do had to be programmed and controlled.

To collect data from the camera a ROS package was used, that extracts the position of diverse body parts of the human body. With that information a program was designed to interpret the movements, based on the different positions that were being read.

The algorithm used for the prediction part is based on machine learning knowledge. A dataset was created to train \textit{Adele}, where all the gestures and movements of the human user are included, as well as the action that the system should take when such movement is acknowledge. To improve the prediction step............................

From the beginning of this project some assumptions were made that represent one of the big foundations of this work. The assumptions made are the following:
\begin{itemize}
\item People tend to interact with other people doing a sequence of gestures
\item The sequence of gestures made in a situation is similar between different people 
\item Multilabel Classification will be able to use information we wouldn't be able to program\\
\qquad $\rightarrow$ This intelligence should improve results over hand programmed solutions
\end{itemize}

% You must have at least 2 lines in the paragraph with the drop letter
% (should never be an issue)


\section{Material and Software Used}
The autonomous system used for this project consisted of a customized \textit{Nomadic Scout} differential drive platform (\emph{Scout}), a  \textit{Katana 300 6M180} robotic arm (\textit{Katana}) and a \textit{RGB-D Kinect} camera.

The Software was all built on top of the Robot \textit{Operating System platform} (ROS) \cite{ROS}, the \textit{Hydro} and \textit{Indigo} versions, and all code was developed in Python 2.7.

For the development of the Dataset and Classifier the Scikit-learn \cite{Scikit-Learn} and the Pandas \cite{Pandas} Python libraries were used.	

\subsection{Kinect}
The \textit{Kinect} was used by the system to collect information about the world, in this case to track the position and movements of the user. 

To extract data from the Kinect we used the OpenNI software which was then connected to the  \textit{openni-tracker} ROS package. This package gives tracks a person and outputs the pose of several joints of the user's body, see Figure \ref{fig:skeleton_tracker}.

\begin{figure}[!h]
	\centering
		\includegraphics[scale=0.5]{./Skeleton_tracker}
	\caption{Points given by the \textit{openni-tracker}}\label{fig:skeleton_tracker}
\end{figure}

\subsection{Katana}
\label{Material: Katana}
The \textit{Katana} arm is the main interaction point between the system and the human, given that the robotic arm has a lot of movement possibilities and was used to replicate human-like behaviors.
To operate the Katana we used the ROS package, \textit{katana\_driver}.  Using this driver it is possible to power off the motors, allowing manual set of the robot positions and a reading of the maximum and minimum values for the different joints. After concluding this step it is possible to send commands to the Katana with the value for each joint position as well as the time to execute the movement. Sending a combination of these commands, it is possible for the Katana to reproduce simple human-like gestures.\\

\subsection{Scout}
The \textit{Scout} was controlled to allow the autonomous system to move in a single dimension to allow the system to approach and move away from the user.
The first layer connecting directly to the Scout robot consisted of the scout\_driver and scout\_odom packages which controlled the velocity of each wheel and the odometry readings.  
Due to difficulties with connecting the Scout with ROS \textit{move\_base} package we decided to program our own controller over the Scout driver. 

Our controler received the commands in ($\delta_x,\delta_y,\delta_\theta$) where each coordinate corresponds to the difference between the goal and the current robot's location. The controller is based on the comparison of the odometry with the distance to the goal, when the robot is within a certain threshold of the goal it stops. We only implemented this controller for one dimension movements, because there were some troubles with the odometry measurements associated with the needed rotation for the two dimensional movement and this didn't restrict too much our setup. Considering that the main goal of the project is focused on the Human robot interaction and not on the robot motion, we decided to leave this point for future refinement of the system.\\

	


%For this project we worked with the robotic arm Katana, the mobile robotic base Scout and the Kinect camera. These 3 components build our system, which we decided to call \textit{Adele}.

%To achieve this goal there are some issues that need to be solved at first, such as: learning how to work with the software ROS (Robot Operating System), making the robot mobile, collecting data from the camera, interpret this data, create an algorithm to make a more human-like response from the robotic system and finally integrate all these parts of the project.

%It is essential to have a good understanding of the ROS software, because in is the foundation of the project. ROS allows to use packages for the robots and camera, as well as creates an interface that facilitates the communication between all the system parts.

%Considering the second point mentioned - making the robot mobile - there is the need to use already created ROS packages that give us drivers both for Scout as for Katana. Furthermore the movements that both robots do had to be programmed and controlled.

%To collect data from the camera a ROS package was used, that extracts the position of diverse body parts of the human body. With that information a program was designed to interpret the movements, based on the different positions that were being read.

%The algorithm used for the prediction part is based on machine learning knowledge. A dataset was created to train \textit{Adele}, where all the gestures and movements of the human user are included, as well as the action that the system should take when such movement is acknowledge. To improve the prediction step............................



%The autonomous system used for this project consisted of a customized \textit{Nomadic Scout} differential drive platform (\emph{Scout}), a  \textit{Katana 300 6M180} robotic arm (\textit{Katana}) and a \textit{Kinect} camera.
%The \textit{Kinect} was used by the system to collect information about the world, in this case to track the position and movements of the user. From the raw camera information an algorithm was implemented to track the user and then detect several gestures. 
%The Scout was controlled to allow the autonomous system to move in a single dimension to allow the system to approach and move away from the user.
%The \textit{Katana} arm is the main interaction point between the system and the human, given that the robotic arm has a lot of movement possibilities and was used to replicate human-like behaviors. 
\section{Our Structure}

\subsection{Kinect Node: Gestures}

\subsection{Action Node: Katana \& Scout Movement}

\subsection{Decision Node: Probabilistic Model}

\section{Conclusion}
For testing our system the user must be aware of the movements that \textit{Adele} is able to recognize. Besides, one must pay attention to the fact that although the processing part of the system is fast, Adele's actions take some time until they are finished. This detail makes the whole interaction feel a bit unnatural.

After many experiments we reach to the conclusion that our implemented system works, because there are many cases where the interaction develops how it is supposed to be. However this didn't happen for every experiment. There are some cases where \textit{Adele} doesn't recognize the correct movement and therefore there is an incorrect reaction, or \textit{Adele} just doesn't recognizes a movement at all and in this case the probability threshold is never reached, which makes Adele stay at the default state, (NOP).

There were several tryouts made where the user makes a defined series of gestures and \textit{Adele}'s recognition and speed of reaction was being tested. This defined series of gestures was defined as:
\begin{enumerate}
\item Waving
\item Calling
\item High-Five
\item Handing
\item Going Back
\end{enumerate}

During the experiments where Adele works as it should, it is difficult to recognize the \textit{prediction} element in the solution. Sometimes the \textit{Adele}'s response time makes it look like it was predicted reaction, but in other cases it seems that \textit{Adele}'s reaction could be explicitly programmed as well. Ignoring this aspect, after several experiments we arrived to the conclusion that Adele has a good performance on 4 of the 5 gestures.

All in all, there are clearly some pros and cons about the implemented solution:

\centerline{Pros}
\begin{itemize}
\item Fast (small processing needed)
\item Scalable - with some improvements it should be able to identify gestures not explicitly programmed
\item Allows the incorporation of both explicit and implicit knowledge (which is impossible with an hand programmed solution)
\end{itemize}

\centerline{Cons}
\begin{itemize}
\item There must exist a training set
\item Hard to evaluate if the robot action is predicted or just a reaction to an user's action
\item Hard to predict performance before complete implementation
\item The features used in this project may not be the best ones
\end{itemize}

To mitigate the disadvantages above mentioned and to improve the general operation of \textit{Adele} there are some improvements that can be made, that will in our opinion affect in a major way \textit{Adele}'s behaviour and emphasize the presence of a "Prediction" element in the implementation.

\begin{itemize}
\item Train Classifier with finer features (for instance, calculate angles between all joints given by the skeleton tracker, or even use all the raw data from the tracker)
\item Train Classifier with features with a constant time interval between samples
\item Incorporate Laser Range Finder, Sonar and Kinect Audio Data
\item Use Kinect to track \textit{Adele}
\item Program the 2-Dimensions controller for Scout.
\end{itemize}

% trigger a \newpage just before the given reference
% number - used to balance the columns on the last page
% adjust value as needed - may need to be readjusted if
% the document is modified later
%\IEEEtriggeratref{8}
% The "triggered" command can be changed if desired:
%\IEEEtriggercmd{\enlargethispage{-5in}}

% references section

% can use a bibliography generated by BibTeX as a .bbl file
% BibTeX documentation can be easily obtained at:
% http://mirror.ctan.org/biblio/bibtex/contrib/doc/
% The IEEEtran BibTeX style support page is at:
% http://www.michaelshell.org/tex/ieeetran/bibtex/
%\bibliographystyle{IEEEtran}
% argument is your BibTeX string definitions and bibliography database(s)
%\bibliography{IEEEabrv,../bib/paper}
%
% <OR> manually copy in the resultant .bbl file
% set second argument of \begin to the number of references
% (used to reserve space for the reference number labels box)
\begin{thebibliography}{1}

\bibitem{Scikit-Learn}
Pedregosa et al., \emph{Scikit-learn: Machine Learning in Python},  JMLR 12, pp. 2825-2830, 2011.

\bibitem{PCA}
Luís B. Almeida \emph{An introduction to principal components analysis}, 2013.

\end{thebibliography}



\end{document}
