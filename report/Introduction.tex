
\section{Introduction}
% no \IEEEPARstart
The main goal of this project is to make a mobile robot display human-aware behaviours, based on the predicted behaviours of the human the robot is interacting with. For this project we worked with the robotic arm Katana, the mobile robotic base Scout and the Kinect camera. These 3 components build our system, which we decided to call \textit{Adele}.

To achieve this goal there are some issues that need to be solved at first, such as: learning how to work with the software ROS (Robot Operating System), making the robot mobile, collecting data from the camera, interpret this data, create an algorithm to make a more human-like response from the robotic system and finally integrate all these parts of the project.

It is essential to have a good understanding of the ROS software, because in is the foundation of the project. ROS allows to use packages for the robots and camera, as well as creates an interface that facilitates the communication between all the system parts.

Considering the second point mentioned - making the robot mobile - there is the need to use already created ROS packages that give us drivers both for Scout as for Katana. Furthermore the movements that both robots do had to be programmed and controlled.

To collect data from the camera a ROS package was used, that extracts the position of diverse body parts of the human body. With that information a program was designed to interpret the movements, based on the different positions that were being read.

The algorithm used for the prediction part is based on machine learning knowledge. A dataset was created to train \textit{Adele}, where all the gestures and movements of the human user are included, as well as the action that the system should take when such movement is acknowledge. To improve the prediction step............................

From the beginning of this project some assumptions were made that represent one of the big foundations of this work. The assumptions made are the following:
\begin{itemize}
\item People tend to interact with other people doing a sequence of gestures
\item The sequence of gestures made in a situation is similar between different people 
\item Multilabel Classification will be able to use information we wouldn't be able to program\\
\qquad $\rightarrow$ This intelligence should improve results over hand programmed solutions
\end{itemize}

% You must have at least 2 lines in the paragraph with the drop letter
% (should never be an issue)
