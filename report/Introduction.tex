
\section{Introduction}
The main goal of this project is to make a mobile robot display human-aware behaviours, based on the predicted behaviours of the human the robot is interacting with. 

% Here insert Theorectical Introduction


The Human Robot Interaction problem was approached with certain assumptions/approximations as a basis for our work. These are the assumptions that our project tries to prove: 

\begin{itemize}
\item People tend to interact with other people with well defined and clear sequences of movements and gestures;
\item The sequences of gestures tend to be similar for a given situation even across different people;
\item A Machine Learning Classifier will be able to capture information that would be too hard to explicitly program. $\rightarrow$ This system should have improved results over explicitly programmed solutions;
\end{itemize}

With these items in mind we developed a setup to test our approach. The robot detected gestures and movements of the user and based on the sequence tried to predict what next action should take.





%For this project we worked with the robotic arm Katana, the mobile robotic base Scout and the Kinect camera. These 3 components build our system, which we decided to call \textit{Adele}.

%To achieve this goal there are some issues that need to be solved at first, such as: learning how to work with the software ROS (Robot Operating System), making the robot mobile, collecting data from the camera, interpret this data, create an algorithm to make a more human-like response from the robotic system and finally integrate all these parts of the project.

%It is essential to have a good understanding of the ROS software, because in is the foundation of the project. ROS allows to use packages for the robots and camera, as well as creates an interface that facilitates the communication between all the system parts.

%Considering the second point mentioned - making the robot mobile - there is the need to use already created ROS packages that give us drivers both for Scout as for Katana. Furthermore the movements that both robots do had to be programmed and controlled.

%To collect data from the camera a ROS package was used, that extracts the position of diverse body parts of the human body. With that information a program was designed to interpret the movements, based on the different positions that were being read.

%The algorithm used for the prediction part is based on machine learning knowledge. A dataset was created to train \textit{Adele}, where all the gestures and movements of the human user are included, as well as the action that the system should take when such movement is acknowledge. To improve the prediction step............................


