\subsection{Kinect Node: Gestures}

The \textit{Sensor/Kinect node} is responsible for processing the raw information of the Kinect and sending structured and useful data to the \textit{decision node}. This is done through the OpenNI software and three ROS nodes, two of them implemented by us.  

From the raw output of the OpenNI software the ROS package \textit{openni\_tracker} outputs the pose of various joints of the user. This data was then fed into our \textit{kinect\_skeleton\_tracker} node. This node obtained a stable representation of the points tracked of the user through filtering the data with a median filter of 5 samples. It also calculated the velocity and position of the center of mass of the user in regards to the camera, through the average of the measurements of the six central points (neck, torso, shoulders and hips). 

From points representing joints it's straightforward to calculate the vectors that represent parts of the body like arm and forearm. Relations between them, such as angles or distances perform a good way to detect movements independent from the camera's referential.
From these angles and distances we can calculate both the values and their derivatives. Extracting these features from the image allowed the detection of both small movements or more complete gestures.

We implemented the detection of gestures of different complexity. The more simple "micro-gestures":

\begin{itemize}
\item forearm opening;
\item forearm closing;
\item arm moving up;
\item arm moving down;
\item arm stopped;
\item hand above elbow;
\item hand under elbow;
\item elbow behind body;
\item elbow in front of body;
\end{itemize}

And the compound gestures:
\begin{itemize}
\item waving;
\item calling;
\item handing;
\end{itemize}

The gestures detected and the information about the position and velocity of the user are then fed to another ROS node implemented by us, the \textit{feature\_publisher}.

This node compiles the information about the last 5 gestures of the user and then publishes it to a ROS topic. This topic is then read by either the decision node or the dataset builder script, this process will be explained further in the report.  