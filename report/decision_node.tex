

\subsection{Decision Node: Probabilistic Model}
As it can be seen in figure \ref{fig:adele_structure} the Decision Node is the central block of this project and where its main focus should be. The role of the Decision Node in this project is to interpret the information coming from the Kinect Node and based on the results of the Training Set make a decision on the command that should be sent to the Action Node.

The Decision Node works together with a Training Set. This Training Set is a set of small videos where one user stands in front of the camera and behaves like if he were in a normal human-human interaction. Some examples of interactions are the following:
\begin{itemize}
\item User moves around; waves; goes towards the camera; hands an object
\item User waves; calls; gives an high-five
\item User calls; goes away
\end{itemize}
For saving the information in the Training Set, the following structure is used:\\

$\big[Distance, [Gesture, Timestamp, Velocity_x]_1,...$\\$
[Gesture, Timestamp, Velocity_x]_5\big]$

To every structure arriving the Training Set, a label is given. This label represents the action that Adele should take as a response to that particular human behaviour.

To explain the idea behind this labelling let's use the waving gesture as an example. The moment immediately before the user starts waving, the information coming from the Kinect is already labelled as a "Waving", so that Adele knows that when the user is at that particular distance and moves his arm at that particular velocity and with that amplitude of movement, he is waving at her and she must react properly (waving back). Ideally the features read by the Kinect node are sufficient to distinguish the "waving" from any other gesture and therefore Adele starts reacting before she sees the complete movement.

As mentioned before, the labels that were given, depend only on the person's movement/gesture and represent Adele's reaction. Table \color{red}\ref{tab:expected_response} \color{black} shows the dependencies between both aspects.
\begin{table}[!h]
\centering
\label{tab:expected_response}
\begin{tabular}{|c|c|}
\hline
\textbf{Person's gesture/movement} & \textbf{Adele's response} \\ \hline
Waving                           & Wave                                 \\
Handing an object                & Grab                            		\\
Calling                          & Go meet user                         \\
Go away                          & Return to home                       \\
High-Five                        & Low-Five                             \\
Low-Five                         & High-Five                            \\
Random                           & NOP                                  \\ \hline
\end{tabular}
\caption{Expected response from Adele}
\end{table}

The next step in the implementation of the Decision Node concerns Adele's learning. The Training Set was divided into two parts, one containing 80\% and the other with the remaining 20\%, the latter being called the validation set.

Regarding how the validation is done, there were some different machine learning algorithms that were tested, such as: SVM, Linear SVM, Logistic Regression, 3 Nearest Neighbours, AdaBoost and Stochastic Gradient Descent. The algorithm that presented the best performance using 10-fold cross-validation was the SVM and therefore that was the chosen one.

There is a probability associated with each action performed by Adele. Based on the user movement she captures using the kinect, she labels each set of 5 gestures detected according with the program previously trained. Each label represents an action to be performed by Adele. To filter the most probable action, based on the user initial movements of a gesture, we set a threshold from which we consider to have a degree of certainty that the action which the probability beats the threshold is the one to be performed. 

Probabilities for each action are incremented based on the number of times Adele labels a set of 5 gestures recognized with that desired action. Furthermore, the intiation of these probabilities is not random. 

Human knowledge was used to help the system to predict which gesture is more probable to be done after a previous gesture. 

Bearing that in mind we designed a probability graph with the probabilities for the sequence of actions we found more natural Adele to perform given previous actions. These probabilities are assigned to each possible action after the degree of certainty about the previous one be enough to beat the threshold. In the beginnig we assume the system is performing the NOP action.

The graph is represented as table showing the probability of doing a particular action after other action. This information is shown in table 3.

\begin{table}[]
\centering
\begin{tabular}{cccccccc}
Previous|Next                        & \textbf{NOP} & \textbf{Wave} & \textbf{H5} & \textbf{L5} & \textbf{Go} & \textbf{Grab} & \textbf{Return} \\ \cline{2-8} 
\multicolumn{1}{c|}{\textbf{NOP}}    & 1/7          & 1/7           & 1/7         & 1/14        & 1/7         & 3/14          & 1/7             \\
\multicolumn{1}{c|}{\textbf{Wave}}   & 1/7          & 0             & 0           & 1/7         & 3/7         & 2/7           & 0               \\
\multicolumn{1}{c|}{\textbf{H5}}     & 1/6          & 1/6           & 0           & 1/6         & 1/6         & 1/6           & 1/6             \\
\multicolumn{1}{c|}{\textbf{L5}}     & 2/7          & 0             & 0           & 0           & 0           & 2/7           & 3/7             \\
\multicolumn{1}{c|}{\textbf{Go}}     & 1/14         & 0             & 1/7         & 1/7         & 0           & 3/7           & 3/14            \\
\multicolumn{1}{c|}{\textbf{Grab}}   & 2/7          & 0             & 1/7         & 1/7         & 0           & 0             & 3/7             \\
\multicolumn{1}{c|}{\textbf{Return}} & 1/6          & 1/6           & 1/6         & 1/6         & 1/6         & 1/6           & 0              
\end{tabular}
\caption{My caption}
\label{my-label}
\end{table}